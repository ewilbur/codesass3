\documentclass[]{article}
\usepackage{amsmath,amsthm,amssymb,amsfonts}
\usepackage{hyperref}

\renewcommand{\Pr}[1]{\text{Pr}\left(#1\right)} % For probability

\let\oldemptyset\emptyset % Rebind the crappy looking emptyset to another variable
\let\emptyset\varnothing % and redfine emptyset as the good looking one

\let\oldphi\phi
\let\phi\varphi 

\newcommand\<{\langle}
\renewcommand\>{\rangle}
\newcommand{\kk}{\ensuremath{\Bbbk}} 
\newcommand{\CC}{\ensuremath{\mathbb{C}}} 
\newcommand{\NN}{\ensuremath{\mathbb{N}}}
\newcommand{\QQ}{\ensuremath{\mathbb{Q}}} 
\newcommand{\RR}{\ensuremath{\mathbb{R}}} 
\newcommand{\ZZ}{\ensuremath{\mathbb{Z}}}
\newcommand{\FF}{\ensuremath{\mathbb{F}}}
\newcommand{\cO}{\ensuremath{\mathcal{O}}} 
\newcommand{\Aut}{\ensuremath{\mathrm{Aut}}} 
\newcommand{\Gal}{\ensuremath{\mathrm{Gal}}} 


\newenvironment{solution}
{
	\begin{proof}[Solution] \text{ }
		\\
	}
	{
	\end{proof}
}

%opening
\title{Assignment 3}
\author{Evan Curry Wilbur}

\begin{document}

\maketitle

\subsection*{5.01} Compute $1 + 1 + 1 + 1 + 1$ in the finite field $\FF_2$.
\begin{solution}
	$1 + 1 + 1 + 1 + 1 = 1$.
\end{solution}
%\begin{solution}
%	$1 + 1 + 1 + 1 + 1 = 1$
%\end{solution}

\subsection*{5.02} Compute
$$
	\underbrace{1 + 1 + 1 + \dots + 1}_\text{107}
$$
in the finite field $\FF_2$.
\begin{solution}
	$$
		\underbrace{1 + 1 + 1 + \dots + 1}_\text{107} = 1.
	$$
\end{solution}

\subsection*{5.03} Compute $\left(1 + x + x^2\right)^4$ as a polynomial with coefficients in $\FF_2$.
\begin{solution}
	Using the \href{https://planetmath.org/freshmansdream}{Freshman's dream} theorem, we are actually allowed to distribute the exponent across the addition so
\begin{align*}
	\left(1 + x + x^2\right)^4 &= 1^4 + x^4 + \left(x^2\right)^4 \\
	&= 1 + x^2 + x^8 
\end{align*}
\end{solution}

\subsection*{5.04} Compute the product $\left(x^4 + x^3 + x^2 + x + 1\right)\left(x^4 + x + 1\right)\left(x^4 + x^3 + 1\right)$ in the collection of polynomials with coefficients in $\FF_2$.
\begin{solution}
	Worked this out on paper. Hopefully you can trust I did the algebra correctly since I'm too lazy to type up all the steps:
	$$
		x^{12}+x^9+x^6+x^3+1
	$$
\end{solution}

\subsection*{5.05} Let $g(x) = x^3 + x + 1$ be a generating polynomials for a CRC/ Compute the CRC for the byte 11100011.
\begin{solution}
	0000
\end{solution}
%\subsection*{5.05} Let $g(x) = x^3 + x + 1$ be a generating polynomial for a CRC. Figure out how to be a little clever in computing the CRC for the bytes
%$$
%	111000110101000110011110
%$$
%so that you don't fill up a whole sheet of paper with an enormously long division.
%\begin{solution}
%	content...
%\end{solution}

\subsection*{5.08} Verify that the CRC with generating polynomial $1 + x^2 + x^3 + x^4$ fails to detect two-bit errors that are a multiple of 7 bits apart.
\begin{solution}
	According to the textbook, it is sufficient to show that $g | x^7 + 1$. Indeed, it can be easily verified that
	$$
		x^7 + 1 = (x^3 + x^2 + 1)(x^4 + x^3 + x^2 + 1)
	$$
\end{solution}

\subsection*{6.01} Factor the integers 1028 and 2057 into primes.
\begin{solution}
	\begin{align*}
		1028 &= 2^2 \times 257 \\
		2057 &= 11^2 \times 17.
	\end{align*}
\end{solution}

\subsection*{6.03} Find the reduction $\mod 88$ of -1000.
\begin{solution}
	$$
		-1000 \cong 56 \mod 88.
	$$
\end{solution}

\subsection*{6.07} Prove in general that if $r$ is the reduction of $N \mod m$, and if $r \neq 0$, then $m - r$ is the reduction of $-N \mod m$.
\begin{solution}
	Let $r \cong N \mod m$ where $0 < r < m$. So there exists a $q \in \ZZ$ such that $N = qm + r$. Then
	\begin{align*}
		N = qm + r \Rightarrow -N &= -qm - r \\
		&= -qm - r + m - m \\
		&= -(q + 1)m + (m - r).
	\end{align*}
	Since $0 < r < m$ it follows that $0 < m - r$. Furthermore, $r > 0$ hence $m - r < m$. Therefore, $0 < m - r < m$ and so $m - r$ is the reduction of $-N$ modulo $m$.
\end{solution}

\subsection*{6.22} Show that for any integer $n$, the integers $n$ and $n^2 + 1$ are relatively prime.
\begin{solution}
	Let $d = \gcd\left(n, n^2 + 1\right)$. Then $d | n$ and $d | n^2 + 1$. But also, $d | n^2$ so it must be that $d | n^2 + 1 - n^2$ since it's just a linear combination of elements that are divisible by $d$. Thus $d | 1$ so $d = 1$.
\end{solution}

\subsection*{6.37} Find $\gcd(1112, 1544)$ and express it in the form $1112x + 1544y$ for some integers $x$ and $y$ by hand computation.
\begin{solution}
	\begin{align*}
		&\gcd(1112, 1544) = 8 \\
		&1112 \times 25 + 1544 \times -18 = 8
	\end{align*}
\end{solution}

\subsection*{6.49} Compute and \textit{reduce modulo} the indicated \textit{modulus:} $110 \times 124 \mod 3$ and also $12 + 1234567890 \mod 10$.
\begin{solution}
	\begin{align*}
		110 \times 124 \mod 3 &\cong 2 \times 1 \mod 3 \\
		&\cong 2 \mod 3
	\end{align*}
	\begin{align*}
		12 + 1234567890 \mod 10 &\cong 2 + 0 \mod 10 \\
		&= 2 \mod 10
	\end{align*}
\end{solution}

\subsection*{6.50} Compute $2^{1000} \% 11$
\begin{solution}
	\begin{align*}
		2^{10} &\cong 1 \mod 11 \\
		\left(2^{10}\right)^{100} &\cong 1^{100} \mod 11 \\
		2^{1000} &\cong 1 \mod 11
	\end{align*}
\end{solution}

\subsection*{6.57} From the definition, find $\phi(36), \phi(18),$ and $\phi(28)$.
\begin{solution}
	\begin{align*}
		\phi(36) &= 12 \\
		\phi(18) &= 6 \\
		\phi(28) &= 12
	\end{align*}
\end{solution}

\subsection*{6.52} Find the multiplicative inverse of 3 modulo 100
\begin{solution}
	$$
		3 \times 67 \cong 1 \mod 100.
	$$
\end{solution}

\subsection*{6.80} Show that $x^2 - y^2 = 102$ has no solution in the integers.
\begin{solution}
	   
\end{solution}

\subsection*{6.81} Show that $x^3 + y^3 = 3$ has no solution in the integers.
\begin{solution}
	Since $x^3 + y^3 = (x + y)(x^2 - xy + y^2)$ and 3 is prime, one of the following two cases would have to be true
	\begin{enumerate}
		\item[1.] $x + y = 1$ and $x^2 - xy + y^2 = 3$
		\item[2.] $x + y = 3$ and $x^2 - xy + y^2 = 1$
	\end{enumerate}
	Suppose the first case. Then $y = 1 - x$ so
	\begin{align*}
		x^2 - xy + y^2 = 3 &\Rightarrow x^2 - x(1 - x) + (1 - x)^2 = 3 \\
		&\Rightarrow 3x^2 - 3x - 2 = 0 \\
		&\Rightarrow x = \frac{3}{2} \text{ or } x = \frac{-1}{2}.
	\end{align*}
	In both of these two solutions, $x$ is not an integer, so case 1 fails. Suppose instead that it is case 2. Then similarly $y = 3 - x$
	\begin{align*}
		x^2 - xy + y^2 = 1 &\Rightarrow x^2 - x(3 - x) + (3 - x)^2 = 1 \\
		&\Rightarrow 3x^2 - 9x + 8 = 0
	\end{align*}
	which has discriminant $-15$, and so has no real solution. In both cases, we are unable to get integer solutions. Thus the equation has no integer solutions.
	
\end{solution}

\subsection*{8.17} Show that
$$
	123456789123456789 + 234567891234567891 \neq 358025680358025680
$$
\begin{solution}
	$$
		123456789123456789 + 234567891234567891 = 358024680358024680
	$$
	and 
	$$
		358024680358024680 \neq 358025680358025680
	$$
\end{solution}

\end{document}
